%%%%%%%%%%%%%%%%%%%%%%%%%%
%
% Rooms Manual - for the HCI Course Project
%
% (Rooms by zeBra & kanGuru)
%
%%%%%%%%%%%%%%%%%%%%%%%%%%

\documentclass{article}

\usepackage{times}

\usepackage[margin=2.0cm]{geometry}

\usepackage{fancyhdr}
	\pagestyle{fancy}
	\lhead{HCI Course Project}
	\chead{\textbf{\LARGE{\textit{Rooms} Manual}} }
	\rhead{by zeBra \& kanGuru}
	\cfoot{\ }

\setlength{\parindent}{0em} 

%			B E G I N
%%%%%%%%%%%%%%%%%%%%%%%%%%
\begin{document}


\section{About \textit{Rooms} }
	The text adventure \textit{Rooms} throws a nameless hero who has an ordinary live,
	into a wicked house that forces him to solve puzzles
	that transcend all reason. Will he be able to escape alive?
	
	In the beginning the hero wakes up in the entrance hall of an old
	dusty shack with nothing but his old pants on and the memories of the latest events lost... \\
	
	To finish the game the user has to solve some major puzzles by commanding the nameless 
	hero through the game by using simple english sentences. Besides following the the main
	target to escape from the house and make it to the \textit{'outside'}, the user may also try a lot of 
	not game solving interactions.
	
	
	
\section{How To Start \textit{Rooms} }
	The game consists of 11 prolog files zipped together in \textit{rooms.zip}. Unzip those into any folder.
	From there call the game directly from the console of your choice with: \\ \\
	 \textbf{\textit{\$ swipl -s rooms.pl -g "start."}  } \\ \\
 	Alternatively start \textbf{ \textit{swipl} } and consult the file \textbf{ \textit{rooms.pl}}. Type: \\ \\
	\textbf{\textit{-? start.}  }
  


\section{Controls}
	After starting the game the first thing the user will see is a short help text that explains how to
	control the game. Generally it is possible to type normal english sentences (use of upper and lower case, or
	sentence signs are free to use but not mandatory). \\
	
	The major condition for the user input is that the instructions always have to start with a verb. Imperatives 
	have to be used like: \\
	
	\begin{tabular}{ l l }
	   \textbf{ \textit{ \textgreater \ go to}} room name & (ex. go to the kitchen) \\
	   \textbf{ \textit{ \textgreater \ look around          }}         &    gives a description of the surroundings             \\
 	   \textbf{ \textit{ \textgreater \ look in }}something &  (ex. look in the desk) \\
  	   \textbf{ \textit{ \textgreater \ look at}} something & (ex. look at the door) \\
 	   \textbf{ \textit{ \textgreater \ take }} something      & (ex. take the apple) \\
 	   \textbf{ \textit{ \textgreater \ show inventory }}            &                  \\
 	   \textbf{ \textit{ \textgreater \ use }}something       &               \\
 	   \textbf{ \textit{ \textgreater \ use }}something  \textbf{ \textit{with}} another thing &    \\
 	   \textbf{ \textit{ \textgreater \ combine }} something \textbf{ \textit{ with}} another thing &  \\
	\end{tabular}\\
	
	Other verbs are also understood: \textbf{ \textit{open, hit, kick, burn, eat, fart, talk, pull, unfold, give, put, jump}} \\ \\
	\textbf{ \textit{'Things'}} are the objects (rooms, items, furniture, etc...) that are described in the game. The whole 
	name of these objects is required. Short cuts like \textit{key} for \textit{rusty key} won't work.
	To exit the game type: \textbf{ \textit{quit, exit, bye}} or \textbf{ \textit{good bye}}.
	
         

\section{In Game Example}
	Here a \textit{standard} game dialogue: \\
	
	\textbf{ \textit{ \textgreater \ Take the shoe!} } \\
	  \hspace*{15 pt}{You have the shoe.} \\
	\hspace*{0 pt}\textbf{ \textit{ \textgreater \ Eat the shoe!} } \\
	{  \hspace*{12 pt}   You're getting hungy and in the shoe you can see the cow that was} \\
 	{  \hspace*{12 pt}   skinned for the leather. You try to eat the shoe, and almost choked} \\
 	{  \hspace*{12 pt}   to death.}  \\


\end{document} 

%			E N D
%%%%%%%%%%%%%%%%%%%%%%%%%%
